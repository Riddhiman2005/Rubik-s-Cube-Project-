
\section{Data Structures and Algorithms}
\label{DSA}
 \textbf{Data Structures} are structures that are used to handle and store data. Data Structures may be presented by default in a programming language, and if they aren't present in there, we can code them in. Having a good knowledge of data structures is essential as it allows the programmer to be able to manipulate the data as needed seamlessly and effortlessly.\par
\vspace{3.5mm}
An \textbf{algorithm} is a finite set of instructions or logic, written in order, to accomplish a certain predefined task. An algorithm is not the complete code or program, it is just the core solution of a problem, which can be expressed either as an informal high-level description using a flowchart.

\subsubsection*{Pseudo-Code}
Because data structures and algorithms are universally applicable to all languages, it doesn't make sense to talk about the logic in terms of a single programming language. Neither can we just simply state the logic as it may not be understood by everyone. \\
Therefore, to express the logic, we use an informal high-level description of the logic. Furthermore, we can make it so that our descriptions follow the general coding syntax, without actually using a programming language.\\
This informal description of the logic in an algorithm is referred to as \textbf{Pseudo-Code}. 'Pseudo' means "False and is called so because though the description looks like a code, but is not.

\subsection*{Tree}
 Trees are used extensively in the representation of data. They consist of a node, and contain subsequent smaller branches which lead downwards \cite{antunes_time_complexity}. 
\vspace{4mm}
\noindent Some terms related to tree structure:
\begin{itemize}
    \item \textbf{Root:} A node is defined as the start of the structure, also known as a root. This is the entry point to the structure.
    \item \textbf{Children:} Immediate node(s) below the current node.
    \item \textbf{Depth:} Length of the path from the root node to that particular node. The depth of a node can be used to determine its position within the tree and to analyze the structure of the tree.
\begin{example}
The root node is considered to be at depth 0. Its children are at depth 1, their children are at depth 2, and so on. 
\end{example}
    \item \textbf{Ancestor:} All the nodes above the current node.
    \item \textbf{Descendants:} All the nodes below the current node
    \item \textbf{Sibling:} Nodes with the same parent
    \item \textbf{Leaf:} Nodes at the end of a particular path are called leaves. They've no children.
\end{itemize}

 Every node has two associated children, one left and one right. One of these is always smaller than or equal to the parent and the other larger, with the same rule applying across the entire tree. Generally, the left node is smaller.


A \textbf{Binary Tree} is a tree in which each node has a maximum of two sub-nodes. A binary tree is very simple, yet is very powerful.


It was a very short discussion of Data Structure and Algorithm. The topic is very vast and I mostly referred a very part from \cite{bds,fox}.
