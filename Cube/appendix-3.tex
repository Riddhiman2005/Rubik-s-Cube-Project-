\section{Some Proofs related to Group Theory}
\label{Cycle Proof}
Here, I've discussed some theorems, propositions, and corollaries that I've used and encountered in the course of the work. I've tried proving them here. Most theorems are well known and found in detail in standard abstract algebra textbooks like \cite{gallian,dummit2003abstract}. 

\begin{theorem}
    Let $G$ be a finite cyclic group of order $n$. Then, any subgroup of $G$ is also cyclic.
\end{theorem}
\begin{proofcustom}
    Since, $G$ is given to be cyclic group of order $n$, so it has a generator $x$.\\
Then, $\boxed{G=\{e,x, x^2, ..., x^{n-1}\}}$

Now, let, H be a subgroup of $G$. If $H=\{e\}$, then H is definitely cyclic.

We, therefore, assume that, $x^k \in H$ for some $k \neq 0$.

Let, $S=\{ m \in \mathbb{N} : x^m \in H\}$
Since, $k\in S$, we've that $S\neq 0$.

By, \textbf{Well Ordering Principle} we know that \textbf{Every nonempty subset of  positive integers has a least element}.



So, S has a least element $d$.

We claim that $H= \langle x^d \rangle$

Now, let $g\in H$ be an arbitrary element. Then, $g= x^u$ for $u \in \mathbb{N}$. \\

Then, by division algorithm, we've 
$u=dq+r$; for $0\leq r \geq d-1$

   $$\boxed{ g= x^u=x^{dq+r}= {(x^d)}^q \cdot x^r \implies x^r= {(x^{dq})}^{-1}g}$$

And, since ${(x^{dq})}^{-1} \in H$ and $g \in H$, So, $x^r \in H$.
But our hypothesis is that d is the smallest natural number such that $x^d \in H$ and thus $r <d$ is a contradiction to this. 
Hence, $r=0$.
Thus, $u=dq \implies d | u$. 
Hence, any element of $H$ is a power of $x^d$.
Consequently, $H= \langle x^d \rangle$ is a cyclic group. 
\end{proofcustom}
\begin{theorem}
    Disjoint cycles commute. 

\end{theorem}

\begin{proofcustom}
    Let us assume $\sigma$ and $\tau$ are two disjoint cycles, that means they've no common elements. 
    We, need to show $\boxed{\sigma\tau = \tau\sigma}$\\
    Let's take 3 elements $\alpha \in \sigma$ and  $\alpha \notin \tau$; $\beta \in \tau$ and  $\beta \notin \sigma$ and $x \notin \tau$ and $\notin \sigma$ \\

    \textbf{Case-I:} \hspace{0.5cm}  $\alpha \in \sigma$ and  $\alpha \notin \tau$, So, $\alpha \notin \tau$, $\tau$ fixes $\alpha$ and doesn't permute to other values [Since the cycles are disjoint, so $\tau$ fixes all elements of $\sigma$], and let's take $\sigma(\alpha) = k$

    $$\sigma\tau=\sigma\tau(\alpha)=\sigma(\alpha)=k$$
    $$\tau\sigma=\tau \sigma(\alpha)=\tau(k)=k
    $$

    And so, $\boxed{\sigma\tau=\tau\sigma}$. \\ 
\textbf{Case-II:}\hspace{0.5cm} $\beta \in \tau$ and  $\beta \notin \sigma$, So, $\beta \notin \sigma$, $\sigma$ fixes $\beta$ and doesn't permute to other values [Since the cycles are disjoint, so $\sigma$ fixes all elements of $\tau$], and let's take $\tau(\beta) = p$
 $$\tau\sigma=\tau \sigma(\beta)=\tau(\beta)=p
    $$
    $$\sigma\tau=\sigma\tau(\beta)=\sigma(p)=p$$
   

    And so, $\boxed{\sigma\tau=\tau\sigma}$. 

   \textbf{Case-III:}\hspace{0.5cm} $x \notin \sigma$ and  $x \notin \tau$, So, both $\tau$, $\sigma$ fixes $x$ and doesn't permute to other values. 
 
    
    $$\sigma\tau=\sigma\tau(x)=\sigma(x)=x$$
   $$\tau\sigma=\tau \sigma(x)=\tau(x)=x$$

    And so, $\boxed{\sigma\tau=\tau\sigma}$.  
    
\end{proofcustom}
\newpage
\begin{theorem}
Every permutation $\pi$ of a finite set $S_n$ can be written as a product of finitely many disjoint cycles.
\end{theorem}
\begin{proofcustom}
    If $\pi$ is the identity then it is a product of zero cycles.

We choose $x \in S_n$ with $\pi\left(x\right) \neq x$ [i.e, $\pi$ is not the identity], and define cycle $\pi_{1}=\left(e, x, \pi\left(x\right), \pi^{2}\left(x\right), \ldots, \pi^{k_{1}-1}\left(x\right)\right)$, where $\pi^{k_{1}}\left(x\right)=x$ [So, $\pi_1$ is a cycle of order $k_1$.].

Now, we take $y$ outside cycle $\pi_{1}$ with $\pi\left(y\right) \neq y$, we then define a second cycle $\pi_{2}=\left(e, y, \pi\left(y\right), \pi^{2}\left(y\right), \ldots, \pi^{k_{2}-1}\left(y\right)\right)$, where $\pi^{k_{2}}\left(y\right)=y$ [So, $\pi_2$ is a cycle of order $k_2$.], and so on, continuing choosing $x_{r+1}$ outside cycles $\pi_{1}, \ldots, \pi_{r}$, with $\pi\left(x_{r+1}\right) \neq x_{r+1}$ until no such $x_{r+1}$ exists. At that stage, we've disjoint cycles $\pi_{1}, \ldots, \pi_{r}$, and for each $x \in S_n$ either $\pi(x)=x$ or $\pi(x)=\pi_{i}(x)$ for some $i$. So, $\pi=\pi_{1} \cdots \pi_{r}$.
\end{proofcustom}

\begin{proposition}
    Any n-cycle can be expressed as a product of $(n - 1)$ 2-cycles.
\end{proposition}

\begin{proofcustom}
    We'll try to prove this by method of induction. 
    For $n=2$, $\sigma= (a_1a_2)$, so for n=2, we've $ 1= (2-1)$ cycle.

    Now, we assume for some $n=k$, the proposition holds, that means, 
    $\sigma=(a_1a_2...a_k)=(a_1a_k)(a_1a_{k-1})...(a_1a_2)$. So, for the $k$ cycle, we can write it as the product of  $k-1$ transpositions. 

    Now, we'll check for $n=k+1$, so, $\sigma=(a_1a_2...a_ka_{k+1})= (a_1a_{k+1})(a_1a_2...a_k)$. Thus, $(a_1a_2...a_k)$ has $k-1$ transpositions and $+1$ for $(a_1a_{k+1})$ transposition, so total $k$ transpositions.

    Thus, the proposition follows. 


\end{proofcustom}
\newpage
\begin{proposition}
    Any permutations in $S_n$, $n>1$ can be written as a product of 2-cycles (transpositions).
\end{proposition}

\begin{proofcustom}
    We can write identity $\in S_n$ as $(1)=(12)(12)$ which is a product of transpositions.

    Also from the previous theorem, we can write every permutation in the form:
 
\begin{align*}
\sigma = &(a_1a_2a_3\dots a_m)(b_1b_2b_3\dots b_r)(c_1c_2c_3\dots c_k)(d_1d_2d_3\dots d_p) \\
&= (a_1a_m)(a_1a_{m-1})\dots(a_1a_2)(b_1b_r)(b_1b_{r-1})\dots(b_1b_2)(c_1c_k)(c_1c_{k-1}) \\
&\quad\dots(c_1c_2)(d_1d_p)(d_1d_{p-1})(d_1d_{p-1})\dots(d_1d_2)
\end{align*}
 It can be seen that, $\sigma$ has been written as a product of transpositions. And so the proposition is proved. 


\end{proofcustom}

\begin{corollary}
    Each $n$-cycle can be written as a product of $(n-1)$ 2-cycles (transpositions). If $n$ is even, the corresponding 2-cycle product has an odd number of 2-cycles, and vice versa.
\end{corollary}

\begin{proofcustom}
    Let, $\sigma= (a_1 \, a_2 \, \ldots \, a_n)$. And since, we know from last proposition that we can decompose any n-cycle into product of (n-1) 2 cycles, So, $\sigma = (a_1 \, a_2)(a_2 \, a_3) \ldots (a_{n-1} \, a_n)(a_n \, a_1)$. Since, there are $(n-1)$ such transpositions in the product, so,
if  $n$ is even, then $(n-1)$ is odd. Therefore, the product of (n-1) transpositions has an odd number of 2-cycles.
If n is odd, then, $(n-1)$ is even. Therefore, the product of $( n-1)$ transposition has an even number of 2-cycles.
\end{proofcustom}